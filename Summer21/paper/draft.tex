% test.tex
\title{A Possible Thin Disk-Wind Mechanism of Broad-line Emission in AGN}

\author{Jason Dexter, Kirk Long -- University of Colorado Boulder}

\newcommand{\abstractText}{\noindent
Following on the work of Chiang and Murray (1996) and Waters et al (2016) we explore new regimes in the so called ``disk-wind" mechanism for explaining the broad-line region in active galactic nuclei (AGN), which requires only the classic thin accretion disk geometry with the addition of a magnetohydrodynamic ``wind" to well match observed spectra. This alternative to the classic ``puffy disk" model yields statistically significant differences in both the assumed size of the underlying accretion disks as well as the masses of the supermassive black holes, motivating a search for future observables that could help identify which class of AGN models are correct. While we can easily obtain similar flux profiles as those observed with our model, we show that the phase profiles still look similar to those we would expect from purely Keplerian rotation of a thin-disk, which matches the latest observations of the phase profile of quasar 3C 273 with the VLTI instrument GRAVITY.
}

%%%%%%%%%%%%%%%%%
% Configuration %
%%%%%%%%%%%%%%%%%

\documentclass[12pt, a4paper]{article}
\usepackage{xurl}
\usepackage{amsmath}
\usepackage[super,comma,sort&compress]{natbib}
\usepackage{abstract}
\renewcommand{\abstractnamefont}{\normalfont\bfseries}
\renewcommand{\abstracttextfont}{\normalfont\small\itshape}
\usepackage{lipsum}
\usepackage{algorithm}
\usepackage{algorithmic}
\usepackage{graphicx}
\usepackage{titling}
\usepackage{float}
\usepackage[compact]{titlesec}
\usepackage[margin=1in]{geometry}

%\setlength{\droptitle}{-1in}
\graphicspath{ {.} }

%%%%%%%%%%%%%%
% References %
%%%%%%%%%%%%%%

% If changing the name of the bib file, change \bibliography{test} at the bottom
\begin{filecontents}{ref.bib}


@article{CM96,
   title={Reverberation Mapping and the Disk-Wind Model of the Broad-Line Region},
   volume={466},
   ISSN={1538-4357},
   url={http://dx.doi.org/10.1086/177543},
   DOI={10.1086/177543},
   journal={The Astrophysical Journal},
   publisher={American Astronomical Society},
   author={Chiang, J. and Murray, N.},
   year={1996},
   month={Aug},
   pages={704}
}

@article{Waters16,
       author = {{Waters}, Tim and {Kashi}, Amit and {Proga}, Daniel and {Eracleous}, Michael and {Barth}, Aaron J. and {Greene}, Jenny},
        title = "{Reverberation Mapping of the Broad Line Region: Application to a Hydrodynamical Line-driven Disk Wind Solution}",
      journal = {\apj},
     keywords = {galaxies: active, galaxies: nuclei, line: profiles, quasars: emission lines, hydrodynamics, Astrophysics - Astrophysics of Galaxies},
         year = 2016,
        month = aug,
       volume = {827},
       number = {1},
          eid = {53},
        pages = {53},
          doi = {10.3847/0004-637X/827/1/53},
archivePrefix = {arXiv},
       eprint = {1601.05181},
 primaryClass = {astro-ph.GA},
       adsurl = {https://ui.adsabs.harvard.edu/abs/2016ApJ...827...53W},
      adsnote = {Provided by the SAO/NASA Astrophysics Data System}
}

@article{GRAVITY,
       author = {{Gravity Collaboration} and {Abuter}, R. and {Accardo}, M. and {Amorim}, A. and {Anugu}, N. and {{\'A}vila}, G. and {Azouaoui}, N. and {Benisty}, M. and {Berger}, J.~P. and {Blind}, N. and {Bonnet}, H. and {Bourget}, P. and {Brandner}, W. and {Brast}, R. and {Buron}, A. and {Burtscher}, L. and {Cassaing}, F. and {Chapron}, F. and {Choquet}, {\'E}. and {Cl{\'e}net}, Y. and {Collin}, C. and {Coud{\'e} Du Foresto}, V. and {de Wit}, W. and {de Zeeuw}, P.~T. and {Deen}, C. and {Delplancke-Str{\"o}bele}, F. and {Dembet}, R. and {Derie}, F. and {Dexter}, J. and {Duvert}, G. and {Ebert}, M. and {Eckart}, A. and {Eisenhauer}, F. and {Esselborn}, M. and {F{\'e}dou}, P. and {Finger}, G. and {Garcia}, P. and {Garcia Dabo}, C.~E. and {Garcia Lopez}, R. and {Gendron}, E. and {Genzel}, R. and {Gillessen}, S. and {Gonte}, F. and {Gordo}, P. and {Grould}, M. and {Gr{\"o}zinger}, U. and {Guieu}, S. and {Haguenauer}, P. and {Hans}, O. and {Haubois}, X. and {Haug}, M. and {Haussmann}, F. and {Henning}, Th. and {Hippler}, S. and {Horrobin}, M. and {Huber}, A. and {Hubert}, Z. and {Hubin}, N. and {Hummel}, C.~A. and {Jakob}, G. and {Janssen}, A. and {Jochum}, L. and {Jocou}, L. and {Kaufer}, A. and {Kellner}, S. and {Kendrew}, S. and {Kern}, L. and {Kervella}, P. and {Kiekebusch}, M. and {Klein}, R. and {Kok}, Y. and {Kolb}, J. and {Kulas}, M. and {Lacour}, S. and {Lapeyr{\`e}re}, V. and {Lazareff}, B. and {Le Bouquin}, J. -B. and {L{\`e}na}, P. and {Lenzen}, R. and {L{\'e}v{\^e}que}, S. and {Lippa}, M. and {Magnard}, Y. and {Mehrgan}, L. and {Mellein}, M. and {M{\'e}rand}, A. and {Moreno-Ventas}, J. and {Moulin}, T. and {M{\"u}ller}, E. and {M{\"u}ller}, F. and {Neumann}, U. and {Oberti}, S. and {Ott}, T. and {Pallanca}, L. and {Panduro}, J. and {Pasquini}, L. and {Paumard}, T. and {Percheron}, I. and {Perraut}, K. and {Perrin}, G. and {Pfl{\"u}ger}, A. and {Pfuhl}, O. and {Phan Duc}, T. and {Plewa}, P.~M. and {Popovic}, D. and {Rabien}, S. and {Ram{\'\i}rez}, A. and {Ramos}, J. and {Rau}, C. and {Riquelme}, M. and {Rohloff}, R. -R. and {Rousset}, G. and {Sanchez-Bermudez}, J. and {Scheithauer}, S. and {Sch{\"o}ller}, M. and {Schuhler}, N. and {Spyromilio}, J. and {Straubmeier}, C. and {Sturm}, E. and {Suarez}, M. and {Tristram}, K.~R.~W. and {Ventura}, N. and {Vincent}, F. and {Waisberg}, I. and {Wank}, I. and {Weber}, J. and {Wieprecht}, E. and {Wiest}, M. and {Wiezorrek}, E. and {Wittkowski}, M. and {Woillez}, J. and {Wolff}, B. and {Yazici}, S. and {Ziegler}, D. and {Zins}, G.},
        title = "{First light for GRAVITY: Phase referencing optical interferometry for the Very Large Telescope Interferometer}",
      journal = {\aap},
     keywords = {instrumentation: interferometers, instrumentation: adaptive optics, Galaxy: center, quasars: emission lines, binaries: symbiotic, stars: pre-main sequence, Astrophysics - Instrumentation and Methods for Astrophysics},
         year = 2017,
        month = jun,
       volume = {602},
          eid = {A94},
        pages = {A94},
          doi = {10.1051/0004-6361/201730838},
archivePrefix = {arXiv},
       eprint = {1705.02345},
 primaryClass = {astro-ph.IM},
       adsurl = {https://ui.adsabs.harvard.edu/abs/2017A&A...602A..94G},
      adsnote = {Provided by the SAO/NASA Astrophysics Data System}
}

@article{Julia,
  author  = "Bezanson, Jeff and Edelman, Alan and Karpinski, Stefan and Shah, Viral B",
  title   = "Julia: A fresh approach to numerical computing",
  year    = "2017",
  journal = "SIAM Review",
  note    = "\url{https://github.com/JuliaLang}"
}

\end{filecontents}

% @misc{LinkReference1,
%   title        = "Link Title",
%   author       = "Link Creator(s)",
%   howpublished = "\url{https://example.com/}",
% }
%
% @misc{Author1,
%   author       = "LastName, FirstName",
%   howpublished = "\url{mailto:email@example.com}",
% }

% Any configuration that should be done before the end of the preamble:
\usepackage{hyperref}
\hypersetup{colorlinks=true, urlcolor=blue, linkcolor=blue, citecolor=blue}

\begin{document}

%%%%%%%%%%%%
% Abstract %
%%%%%%%%%%%%

% \twocolumn[
%   \begin{@twocolumnfalse}
%     \maketitle
%     \begin{abstract}
%       \abstractText
%       \newline
%       \newline
%     \end{abstract}
%   \end{@twocolumnfalse}
% ]

\maketitle
\begin{abstract}
  \abstractText
  \newline
  \newline
\end{abstract}

%%%%%%%%%%%
% Article %
%%%%%%%%%%%

\section{Background}
One of the most fascinating discoveries of the last century was the observation that quasars show a remarkable degree of Doppler line-broadening, expanding line-widths to the order of thousands of kilometers per second under the influence of the extreme gravity of the central supermassive black hole. The extremely luminous nature of most quasars makes them a good candidate for $\sim$ thin-disk accretion models, but a thin-disk under ordered rotation should produce a Doppler broadened line profile that is double peaked, which is not observed. The line profile can be adequately explained by instead assuming a puffy ``cloud" model, where almost all quasars are seen at low inclinations and random chaotic motions of parcels of gas create the single-peaked profiles that are observed. While this works to explain the line emission in those conditions, such conditions are relatively restrictive \textemdash it's harder to explain the large luminosity this way \textemdash and there is no further evidence that this is what actually occurs around quasars. With the advent of interferometry we can constrain the centroids and phase of the line emission, and we see in data collected by GRAVITY that there is a clear thin-disk like phase profile around quasar 3C 273. In 1996 Chiang and Murray published an alternative explanation that could create the observed single-peak line profile at high inclinations with just a thin-disk and a magnetohydrodynamical ``wind", a conceptual picture that seems more in keeping with the dynamics of the accretion disk and doesn't rely on complicated and as yet unobserved chaotic clouds. We extend this work to all inclinations, and present best fit parameters for our model to the quasar 3C 273, which most importantly result in a statistically significant difference in assumed black hole mass from the ``cloud" model.

\section{Methods}
Following from the $\hat{n}$ given in Waters and the rate of strain tensor terms in spherical coordinates from Batchelor we verify and expand the result given in CM96:

\begin{equation} \label{eq1}
\begin{split}
\hat{n}\cdot\boldsymbol{\Lambda}\cdot\hat{n} = & \sin^2i\left[\frac{\partial v_r}{\partial r}\sin^2\phi +
\left(\frac{\partial v_\phi}{\partial r} - \frac{v_\phi}{r}\right)\sin\phi\cos\phi + \frac{v_r}{r}\cos^2\phi \right] \\
& -\sin i\cos i\left[\left(\frac{1}{r}\frac{\partial v_r}{\partial \theta} + \frac{\partial v_\theta}{\partial r} - \frac{v_\theta}{r}\right)\sin\phi  + \frac{1}{r}\frac{\partial v_\phi}{\partial \theta}\cos\phi\right] \\
& +\cos^2i\left(\frac{1}{r}\frac{\partial v_\theta}{\partial \theta} + \frac{v_r}{r}\right)
\end{split}
\end{equation}

Where in arriving at the form above we have assumed all of the $\frac{\partial}{\partial \phi}$ operator terms are 0 (axisymmetric) and the disk is in the equatorial plane ($\theta = \frac{\pi}{2}$) which allows us to significantly simplify $\hat{n} = \left(\sin\theta\cos\phi\sin i + \cos\theta\cos i\right)\hat{r} + \left(\cos\theta\cos\phi\sin i - \sin\theta\cos i\right)\hat{\theta} - \left(\sin\phi\sin i\right)\hat{\phi}$. Chiang and Murray also immediately discarded the $\theta$ terms, but we keep them here to generalize the wind in both the ``vertical" and radial directions. Waters uses a $\phi$ convention that differs from CM96 by $-\frac{\pi}{2}$, and applying this to equation one gives us:

\begin{equation} \label{eq2}
\begin{split}
\hat{n}\cdot\boldsymbol{\Lambda}\cdot\hat{n} = & \sin^2i\left[\frac{\partial v_r}{\partial r}\cos^2\phi - \left(\frac{\partial v_\phi}{\partial r} - \frac{v_\phi}{r}\right)\sin\phi\cos\phi + \frac{v_r}{r}\sin^2\phi \right] \\
& -\sin i\cos i\left[\left(\frac{1}{r}\frac{\partial v_r}{\partial \theta} + \frac{\partial v_\theta}{\partial r} - \frac{v_\theta}{r}\right) \cos\phi - \frac{1}{r}\frac{\partial v_\phi}{\partial \theta}\sin\phi\right] \\
& +\cos^2i\left(\frac{1}{r}\frac{\partial v_\theta}{\partial \theta} + \frac{v_r}{r}\right)
\end{split}
\end{equation}

In CM96 they assume that $v_r \approx 0$ (thin-disk model) but that there is an acceleration related to the escape velocity, ie $\frac{\partial v_r}{\partial r} \approx 3\sqrt{2}\frac{v_\phi}{r}$, where $v_\phi = \sqrt{\frac{GM}{r}}$ is the Keplerian $v_\phi$, which gives us $\frac{\partial v_\phi}{\partial r} =  \frac{-v_\phi}{2r}$. These accelerations are important, as we assume the disk is optically thick, and in this case a test particle can quickly accelerate and collide, ``resonating" with the optically thick medium and producing significant emission without signficant motion. (Cite sobolyev or whatever paper).

But what are the $\theta$ terms? Following in the footsteps of CM96 it makes sense to assume that on average $v_\theta \approx 0$ for the same reason $v_r \approx 0$, but similarly we will assume a particle may be lifted by the wind and accelerated to the local escape velocity (but now in the $\hat{\theta}$ direction) such that $\frac{\partial v_\theta}{\partial \theta} \approx \frac{v_{esc}}{\left(H/R\right)}$ and $\frac{\partial v_\theta}{\partial r} \approx \frac{\partial v_r}{\partial r}$. Since $v_\phi(r) \rightarrow \frac{\partial v_\phi}{\partial v_\theta} = 0$, and we also set $\frac{\partial v_r}{\partial \theta} = 0$.

Plugging in these approximations reduces equation two to:

\begin{equation} \label{eq3}
\begin{split}
\hat{n}\cdot\boldsymbol{\Lambda}\cdot\hat{n} = & 3\frac{v_\phi}{r} \sin^2i\cos\phi\left[\sqrt{2}\cos\phi + \frac{\sin\phi}{2}\right]\\
& -\sin i \cos i \left[3\sqrt{2}\frac{v_\phi}{r}\cos\phi\right] \\
& +\cos^2i\left(\frac{1}{r}\frac{v_{esc}}{(H/R)}\right)
\end{split}
\end{equation}

Rescaling $v_\phi$ into units of $r_s$ gives us $v_\phi = \sqrt{\frac{1}{2r'}}$ (where $r' = r/r_s$ and the extra factor of c is just absorbed into an overall normalizing constant since we normalize by the maximum flux in fitting). Similarly the $H/R$ dependence can be absorbed into an overall constant, and we can convert all our $1/r$ terms to be in terms of $r_s$ and absorb the extra scaling terms into our normalization, so that we get an equation that just shows us the $r$ dependence of the line of sight velocity gradient:

\begin{equation} \label{eq4}
\begin{split}
\hat{n}\cdot\boldsymbol{\Lambda}\cdot\hat{n} \approx \frac{\mathrm{d}v_l}{\mathrm{d}l} \approx & 3\frac{\sqrt{\frac{1}{2r}}}{r} f_1 \sin^2i\cos\phi\left[\sqrt{2}\cos\phi + \frac{\sin\phi}{2}\right]\\
& - f_2 \sin i \cos i \left[3\sqrt{2}\frac{\sqrt{\frac{1}{2r}}}{r}\cos\phi\right] \\
& + f_3 \cos^2i\frac{1}{\sqrt{r^3}}
\end{split}
\end{equation}

This is the form used in the fitting routine in the code, where $f_{1,2,3}$ are constants we fit to to determine the overall importance of each term in the wind. The inclination dependence makes the $\cos^2i$ term very important at low inclinations (which was not previously explored), and the addition of both terms at moderate inclinations make a significant difference when compared with the results shown in CM96. It's also interesting that the $\cos^2i$ term has no $\phi$ dependence.

\section{Results and Discussion}


\section{Conclusions}


%%%%%%%%%%%%%%
% References %
%%%%%%%%%%%%%%

\nocite{*}
\bibliographystyle{plain}
\bibliography{ref}


\end{document}

% Create PDF on Linux:
% FILE=test; pkill -9 -f ${FILE} &>/dev/null; rm -f ${FILE}*aux ${FILE}*bbl ${FILE}*
